\documentclass[UTF8]{ctexart}

\usepackage{WeeklyReport}

\title{周报01}
\author{Jeff Fu}
\date{\today}

\begin{document}
    \maketitle
    % \tableofcontents
    \section{学习内容}
        \begin{itemize}
            \item Multi-Source DA相关代码分析和学习(主要关注Loss)
        \end{itemize}
    \section{学习收获}
        学习论文中出现的Loss的实现
        \subsection{Moment Matching}
            参考文章Moment Matching for Multi-Source Domain Adaptation的代码
            \subsubsection{MMD}
                $$
                    MMD^2(X, Y) = \left\| {\frac{1}{M}\sum\limits_{i = 1}^M {\phi ({\rm{x}}_i) - } \frac{1}{N}\sum\limits_{j = 1}^N {\phi ({\rm{y}}_j)} } \right\|_\mathcal{H}^2
                $$
                \begin{verbatim}
    def linear_mmd2(fx, fy):
        loss = 0.0
        delta = fx - fy
        # 为什么存在错位?
        loss = torch.mean((delta[0:-1] * delta[1:]).sum(1))
        return loss

    def poly_mmd2(fx, fy, d=2, alpha=1.0, c=2.0)
        # kernel trick?
        K_XX = (alpha * (fx[:-1] * fx[1:]).sum(1) + c)
        K_XX_mean = torch.mean(K_XX.pow(d))
    
        K_YY = (alpha * (fy[:-1] * fy[1:]).sum(1) + c)
        K_YY_mean = torch.mean(K_YY.pow(d))
    
        K_XY = (alpha * (fx[:-1] * fy[1:]).sum(1) + c)
        K_XY_mean = torch.mean(K_XY.pow(d))
    
        K_YX = (alpha * (fy[:-1] * fx[1:]).sum(1) + c)
        K_YX_mean = torch.mean(K_YX.pow(d))
    
        return K_XX_mean + K_YY_mean - K_XY_mean - K_YX_mean
                \end{verbatim}
            \subsubsection{K阶矩}
                文章代码实现的k\_moment跟概率论中的K阶矩貌似有些差别?
                \begin{verbatim}
    # 欧氏距离
    def L2(x1, x2):
        # x1和x2是n维向量
        return ((x1-x2)**2).sum().sqrt()
    
    # k moment
    def k_moment(s1, s2, s3, s4, t, k):
        # 固定源域的数量?(4)
        s1 = (s1**k).mean(0)
        s2 = (s2**k).mean(0)
        s3 = (s3**k).mean(0)
        # 为什么没有s4?(按论文公式应该需要下面这行)
        s4 = (s4**k).mean(0)
        t = (t**k).mean(0)

        # 两两组合构成k阶矩(按论文公式这里应该还需要除以N,但这里没有除N)
        return L2(s1, t) + L2(s2, t) + L2(s3, t) + \
            L2(s1, s2) + L2(s2, s3) + L2(s3, s1) + \
            L2(s4, s1) + L2(s4, s2) + L2(s4, s2) + \
            L2(s4, t)
                \end{verbatim}

                代码中还实现了一个regularizer(用于规范化?)
                \begin{verbatim}
    def msda_regularizer(s1, s2, s3, s4, t, belta_moment):
        print('s1:{}, s2:{}, s3:{}, t:{}'.format(s1.shape, s2.shape, s3.shape, t.shape))
        # s1, s2, s3, t: (128, 2048)
        # 在这里仍然只用了三个Source?
        s1_mean = s1.mean(0)
        s2_mean = s2.mean(0)
        s3_mean = s3.mean(0)
        t_mean = t.mean(0)
        s1 = s1 - s1_mean
        s2 = s2 - s2_mean
        s3 = s3 - s3_mean
        t = t - t_mean
        moment1 = L2(s1, t) + L2(s2, t) + L2(s3, t) + \
            L2(s1, s2) + L2(s2, s3) + L2(s3, s1) + \
            L2(s4, s1) + L2(s4, s2) + L2(s4, s2) + \
            L2(s4, t)
        reg_info = moment1
        # print(reg_info)
        for i in range(belta_moment - 1):
            # s1, s2, s3, s4, t, k
            reg_info += k_moment(s1, s2, s3, s4, t, i+2)

        return reg_info/6
        # return L2(s1, t)
                \end{verbatim}
            \subsubsection{数据处理}
                Digit-Five数据集比较方便获得,所以先尝试用Digit-Five进行测试

                下载后得到的是已经预处理过的mat文件
                \begin{verbatim}
    # 读取数据(以mnist为例)
    def load_mnist(scale=True):
        mnist_data = loadmat(base_dir + '/mnist_data.mat')
        """
        raw data的shape如下
        train_28 (55000, 28, 28, 1)
        test_28 (10000, 28, 28, 1)
        train_32 (55000, 32, 32)
        test_32 (10000, 32, 32)
        label_train (55000, 10)
        label_test (10000, 10)
        """
        if scale:   # 32x32
            mnist_train = np.reshape(mnist_data['train_32'], (55000, 32, 32, 1))
            mnist_test = np.reshape(mnist_data['test_32'], (10000, 32, 32, 1))
            mnist_train = np.concatenate([mnist_train, mnist_train, mnist_train], 3)
            mnist_test = np.concatenate([mnist_test, mnist_test, mnist_test], 3)
            mnist_train = mnist_train.transpose(0, 3, 1, 2).astype(np.float32)
            mnist_test = mnist_test.transpose(0, 3, 1, 2).astype(np.float32)
            mnist_labels_train = mnist_data['label_train']
            mnist_labels_test = mnist_data['label_test']
        else:   # 28x28
            mnist_train = mnist_data['train_28']
            mnist_test =  mnist_data['test_28']
            mnist_labels_train = mnist_data['label_train']
            mnist_labels_test = mnist_data['label_test']
            mnist_train = mnist_train.astype(np.float32)
            mnist_test = mnist_test.astype(np.float32)
            mnist_train = mnist_train.transpose((0, 3, 1, 2))
            mnist_test = mnist_test.transpose((0, 3, 1, 2))
        train_label = np.argmax(mnist_labels_train, axis=1)
        # 随机排序
        inds = np.random.permutation(mnist_train.shape[0])
        mnist_train = mnist_train[inds]
        train_label = train_label[inds]
        test_label = np.argmax(mnist_labels_test, axis=1)
    
        mnist_train = mnist_train[:25000]
        train_label = train_label[:25000]
        mnist_test = mnist_test[:25000]
        test_label = test_label[:25000]
        print('mnist train X shape->', mnist_train.shape)   # (25000, 3, 32, 32)
        print('mnist train y shape->', train_label.shape)   # (25000,)
        print('mnist test X shape->', mnist_test.shape)     # (10000, 3, 32, 32)
        print('mnist test y shape->', test_label.shape)     # (10000,)

        return mnist_train, train_label, mnist_test, test_label
                \end{verbatim}

                用同样方法读取其它domain的data,测试得到其它domain的shape如下
                \begin{lstlisting}
    # mnist_m
    train (55000, 28, 28, 3)
    test (10000, 28, 28, 3)
    label_train (55000, 10)
    label_test (10000, 10)
    mnist_m train X shape-> (25000, 3, 28, 28)
    mnist_m train y shape-> (25000,)
    mnist_m test X shape-> (9000, 3, 28, 28)
    mnist_m test y shape-> (9000,)
    # usps
    img_train (7438, 1, 28, 28)
    label_train (7438, 1)
    img_test (1860, 1, 28, 28)
    label_test (1860, 1)
    usps train X shape-> (29752, 1, 28, 28)
    usps train y shape-> (29752,)
    usps test X shape-> (1860, 1, 28, 28)
    usps test y shape-> (1860,)
    # synth_number
    syn_train (25000, 32, 32, 3)
    syn_test (9000, 32, 32, 3)
    syn_label_train (25000, 1)
    syn_label_test (9000, 1)
    syn number train X shape-> (25000, 3, 32, 32)
    syn number train y shape-> (25000,)
    syn number test X shape-> (9000, 3, 32, 32)
    syn number test y shape-> (9000,)
    # svhn
    syhn_train_img (32, 32, 3, 73257)
    svhn_train_label (73257, 1)
    svhn_test_img (32, 32, 3, 26032)
    svhn_test_label (26032, 1)
    svhn train X shape-> (25000, 3, 32, 32)
    svhn train y shape-> (25000,)
    svhn test X shape-> (9000, 3, 32, 32)
    svhn test y shape-> (9000,)
                \end{lstlisting}
            \subsubsection{试运行}
                经过一些修改之后,将模型跑通了,在训练前期可以观察到Loss的下降,一段时间后出现
                BrokenPipeError
                (查资料显示可能原因在于Windows环境中的多线程处理问题,而本地Linux虚拟机不支持CUDA,所以没有进一步测试)
                \begin{lstlisting}
    Train Epoch: 0 [0/100 (0%)] Loss1: 5.847885  Loss2: 5.929170 Discrepancy: 0.028762
    Train Epoch: 0 [100/100 (0%)] Loss1: 0.640006  Loss2: 0.655498 Discrepancy: 0.004664
    BrokenPipeError: [Errno 32] Broken pipe
                \end{lstlisting}
    \section{疑问/困难}
        \begin{enumerate}
            \item 某些代码的实现与原文有些出入,按照自己的理解进行了修改
            \item 训练的时候出现BrokenPipeError,可能Windows多线程处理出现问题
        \end{enumerate}
\end{document}
