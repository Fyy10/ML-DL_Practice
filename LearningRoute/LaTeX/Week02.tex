\documentclass[UTF8]{ctexart}

\usepackage{WeeklyReport}

\title{周报02}
\author{Jeff Fu}
\date{\today}

\begin{document}
    \maketitle
    % \tableofcontents
    \section{学习内容}
        \begin{itemize}
            \item 模仿Multi-Source DA的代码,实现对应的模型
        \end{itemize}
    \section{学习收获}
        仍然是Moment Matching for Multi-Source Domain Adaptation这篇文章,
        上周学习了Loss和数据读取的部分,这周把模型和优化的部分看懂,然后模仿着实现这个模型。

        为了方便在本地进行测试,针对Windows环境进行了一些修改,现在已经可以完整地跑完一个epoch。

        代码中的C1和C2分别表示论文中的C和C',而不同source的classifier其实是share weight的(原文好像没有提这一点)
        \subsection{Model}
            M3SDA模型的实现。
            \subsubsection{Argparser}
                在\text{main.py}中,使用argparser进行模型总体参数的设置。
                \begin{itemize}
                    \item record\_folder: 记录文件的保存地址
                    \item batch\_size
                    \item checkpoint\_dir: 模型的保存地址
                    \item optimizer: 选择优化器
                    \item source: 选择的源域(从代码上看,这个参数并没有用上)
                    \item target: 选择的目标域
                    \item eval\_only: 进行评估
                \end{itemize}

                代码中的argparse的实现没有什么特别的,所以没有列出来。
            \subsubsection{Solver}
                在文中有提到,M3SDA的训练涉及到一个feature extractor(在代码中用generator G表示)
                和两个classifier(分别为C和C',在代码中是C1和C2)。G需要减小discrepancy而C'需要增大discrepancy。

                训练时所使用的优化公式为(代码中并没有使用$\lambda$):
                $$
                \min_{G,\mathcal C}\sum_{i=1}^{N}\mathcal L_{\mathcal D_i} + \lambda \min_{G}MD^2(\mathcal D_S,\mathcal D_T)
                $$

                $$
                \min_{\mathcal C'}\sum_{i=1}^{N}\mathcal L_{\mathcal D_i} - \sum_{i}^N |P_{C_i}(D_T)-P_{C_i'}(D_T)|
                $$

                $$
                \min_{G} \sum_{i}^N |P_{C_i}(D_T)-P_{C_i'}(D_T)|
                $$

                模型的主要代码如下(对代码的理解写成了注释):
                \begin{verbatim}
    def feat_all_domain(self, img_s1, img_s2, img_s3, img_s4, img_t):
        return self.G(img_s1), self.G(img_s2), self.G(img_s3), \
            self.G(img_s4), self.G(img_t)

    def C1_all_domain(self, feat_s1, feat_s2, feat_s3, feat_s4, feat_t):
        return self.C1(feat_s1), self.C1(feat_s2), self.C1(feat_s3), \
            self.C1(feat_s4), self.C1(feat_t)

    def C2_all_domain(self, feat_s1, feat_s2, feat_s3, feat_s4, feat_t):
        return self.C2(feat_s1), self.C2(feat_s2), self.C2(feat_s3), \
            self.C2(feat_s4), self.C2(feat_t)

    # returns a tuple of loss
    # out: output, l: label
    def softmax_loss_all_domain(self, out1, out2, out3, out4, l1, l2, l3, l4):
        # criterion
        L = nn.CrossEntropyLoss().cuda()
        return L(out1, l1), L(out2, l2), L(out3, l3), L(out4, l4)

    # overall loss
    # i: image, l: label
    def loss_all_domain(self, i1, i2, i3, i4, it, l1, l2, l3, l4):
        # feature
        f1, f2, f3, f4, ft = self.feat_all_domain(i1, i2, i3, i4, it)
        # out c1
        o1c1, o2c1, o3c1, o4c1, otc1 = self.C1_all_domain(f1, f2, f3, f4, ft)
        # out c2
        o1c2, o2c2, o3c2, o4c2, otc2 = self.C2_all_domain(f1, f2, f3, f4, ft)
        loss_msda = 5e-4 * msda.msda_regulizer(f1, f2, f3, f4, ft, 5)
        # loss
        loss_s1_c1, loss_s2_c1, loss_s3_c1, loss_s4_c1 = self.softmax_loss_all_domain(
            o1c1, o2c1, o3c1, o4c1, l1, l2, l3, l4
        )
        loss_s1_c2, loss_s2_c2, loss_s3_c2, loss_s4_c2 = self.softmax_loss_all_domain(
            o1c2, o2c2, o3c2, o4c2, l1, l2, l3, l4
        )
        return loss_s1_c1, loss_s2_c1, loss_s3_c1, loss_s4_c1, \
            loss_s1_c2, loss_s2_c2, loss_s3_c2, loss_s4_c2, loss_msda

    def train_MSDA(self, epoch, record_file=None):
        criterion = nn.CrossEntropyLoss()
        self.G.train()
        self.C1.train()
        self.C2.train()
        # 设置随机数种子
        torch.cuda.manual_seed(1)

        for batch_idx, data in enumerate(self.datasets):
            # 取数据
            img_t = data['T'].cuda()
            img_s1 = data['S1'].cuda()
            img_s2 = data['S2'].cuda()
            img_s3 = data['S3'].cuda()
            img_s4 = data['S4'].cuda()
            label_s1 = data['S1_label'].long().cuda()
            label_s2 = data['S2_label'].long().cuda()
            label_s3 = data['S3_label'].long().cuda()
            label_s4 = data['S4_label'].long().cuda()

            # 判断最后一部分不构成batch的数据
            if img_s1.size(0) < self.batch_size or img_s2.size(0) < self.batch_size or \
                    img_s3.size(0) < self.batch_size or img_s4.size(0) < self.batch_size or \
                    img_t.size(0) < self.batch_size:
                break

            # zero grad
            self.reset_grad()

            # loss c1 c2 msda
            loss_s1_c1, loss_s2_c1, loss_s3_c1, loss_s4_c1, \
                loss_s1_c2, loss_s2_c2, loss_s3_c2, loss_s4_c2, loss_msda =\
                self.loss_all_domain(img_s1, img_s2, img_s3, img_s4, img_t, \
                abel_s1, label_s2, label_s3, label_s4)

            # loss source
            loss_s_c1 = loss_s1_c1 + loss_s2_c1 + loss_s3_c1 + loss_s4_c1
            loss_s_c2 = loss_s1_c2 + loss_s2_c2 + loss_s3_c2 + loss_s4_c2

            # over all loss
            # 这个地方在原文中是加权相加,而代码中貌似没有体现加权
            loss = loss_s_c1 + loss_s_c2 + loss_msda

            loss.backward()
            
            # 整体的优化
            self.opt_g.step()
            self.opt_c1.step()
            self.opt_c2.step()

            # classifier要增大discrepancy,feature extractor要减小discrepancy
            self.reset_grad()
            loss_s1_c1, loss_s2_c1, loss_s3_c1, loss_s4_c1, \
                loss_s1_c2, loss_s2_c2, loss_s3_c2, loss_s4_c2, loss_msda =\
                self.loss_all_domain(img_s1, img_s2, img_s3, img_s4, img_t, \
                label_s1, label_s2, label_s3, label_s4)

            feature_t = self.G(img_t)
            out_t1 = self.C1(feature_t)
            out_t2 = self.C2(feature_t)

            # loss source
            loss_s_c1 = loss_s1_c1 + loss_s2_c1 + loss_s3_c1 + loss_s4_c1
            loss_s_c2 = loss_s1_c2 + loss_s2_c2 + loss_s3_c2 + loss_s4_c2

            # over all loss
            loss_s = loss_s_c1 + loss_s_c2 + loss_msda

            # discrepancy
            loss_dis = self.discrepancy(out_t1, out_t2)
            loss = loss_s - loss_dis
            loss.backward()
            # 固定G,优化C,增大discrepancy
            # self.opt_g.step()
            self.opt_c1.step()
            self.opt_c2.step()

            # 这里循环4次的意义不知道在哪里……
            for i in range(4):
                self.reset_grad()
                feature_t = self.G(img_t)
                out_t1 = self.C1(feature_t)
                out_t2 = self.C2(feature_t)
                loss_dis = self.discrepancy(out_t1, out_t2)
                loss_dis.backward()
                # 固定C,优化G,减小discrepancy
                self.opt_g.step()
        return batch_idx
                \end{verbatim}

                这里的G,C1和C2都是比较常规的架构(G使用三层CNN和两层FC,C使用一层FC)

                代码中也实现了梯度反转层,但是在模型的forward的过程中并没有使用,目前对这一块的理解还不到位

                \begin{verbatim}
    from torch.autograd import Function

    # 梯度反转层
    class GradReverse(Function):
        def __init__(self, lambd):
            self.lambd = lambd

        def forward(self, x):
            # 为什么不直接返回x?(因为需要x参加计算?)
            return x.view_as(x)

        def backward(self, grad_output):
            return grad_output * -self.lambd

    def grad_reverse(x, lambd=1.0):
        return GradReverse(lambd)(x)
                \end{verbatim}}
                
                至于测试部分,代码跟train比较像,只是没有使用C2(C')。
        \subsection{试运行}
            将代码的其余部分(数据读取等)补充完整,试着运行了一下,
            这次解决了之前多线程处理的问题,可以完整地跑完epoch和测试。

            如果把svhn作为target,把mnist、mnistm、usps、syn作为source,
            训练几个epoch,可以看到测试的正确率在上升,模型收敛的速度比较快,
            可能是因为在原代码中的默认参数设置得比较好。
            \begin{lstlisting}
    python main.py --target svhn
    Dataset Loading...
    ......
    Load Finished!
    Model Loaded (1 G 2 C)!
    Initialization Completed.

    0
    Train Epoch: 0 [0/100 (0%)]     Loss1: 5.880786  Loss2: 5.789978    Discrepancy: 0.029394
    Train Epoch: 0 [100/100 (0%)]   Loss1: 0.574418  Loss2: 0.574782    Discrepancy: 0.004905
    Test set: Average loss: -0.0319, Accuracy C1: 6201/9000 (69%)

    1
    Train Epoch: 1 [0/100 (0%)]     Loss1: 0.246108  Loss2: 0.243026    Discrepancy: 0.003677
    Train Epoch: 1 [100/100 (0%)]   Loss1: 0.338004  Loss2: 0.333200    Discrepancy: 0.002567
    Test set: Average loss: -0.0305, Accuracy C1: 6573/9000 (73%)

    2
    Train Epoch: 2 [0/100 (0%)]     Loss1: 0.270224  Loss2: 0.264725    Discrepancy: 0.002441
    Train Epoch: 2 [100/100 (0%)]   Loss1: 0.290649  Loss2: 0.292531    Discrepancy: 0.002519
    Test set: Average loss: -0.0304, Accuracy C1: 6625/9000 (74%)
            \end{lstlisting}

            可以看到,经过3个epoch后,正确率达到了74\%左右,
            考虑到时间原因,没有继续往下训练。
    \section{疑问/困难}
        \begin{enumerate}
            \item 和之前一样,代码中的实现跟论文的描述有些区别
            \item 感觉上有一些代码是“冗余”的,在训练和测试的时候并没有用到
            \item 有不止一篇文章提到了梯度反转层(这次看的代码里也有体现),但目前对这个的作用还不太理解
        \end{enumerate}
\end{document}
