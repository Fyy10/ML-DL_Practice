\documentclass[UTF8]{ctexart}

\usepackage{WeeklyReport}
\usepackage{graphicx}
\usepackage{amsmath}
\usepackage{amssymb}

\title{周报12}
\author{傅阳烨}
\date{\today}

\begin{document}
\maketitle
% \tableofcontents
\section{本周计划}
\begin{itemize}
    \item proposal思路梳理,method简述
    \item 实验进度及结果
\end{itemize}

\section{Proposal}

在多源领域适应中,假设所有 source domain 的并集能够包含 target 的所有信息,而 target 的特征只是其中一部分,
强行把 target 的特征与整个 feature map 进行对齐反而会影响分类器的判断,所以提出只进行部分特征的对齐。

对于不同的类别和域,期望在选择后的特征图上,保持源域与目标域接近,域的类别内数据聚集,而不同类别之间需要有一定差异,因此提出类别分布损失。

一些与数据的中心点差距过大的样本,对训练有干扰,训练的时候跳过,不进行参数更新。

\subsection{Method}

经过特征提取,得到 feature map,在 feature map 上使用一个部分特征对齐的网络,得到提纯后的特征(选取 target 所具有的特征),对于 target 则不需要进行特征选择。

特征提取器提取的特征需要是容易分辨的,且相同的类别应当比较靠近,用同一类别的特征平均值当作中心点,用流动平均值维护中心点,target 的类别由 pseudo label 确定,
用 loss 限制相同类别的数据在提纯后的特征上靠近,不同类别的数据的中心点相互分离。对于偏离中心点过多的数据,设置一个 threshold,threshold 由 target 与 source 的相关性得到,距离高于 threshold 的样本不进行参数 update,threshold 也用流动平均值去维护。

由于分类器早期产生的 pseudo label 不可靠,与 target label 相关的 loss 所占权重在训练过程中从0开始逐渐增加(可使用 arctan 等函数)。

在假设情况下,分类器所看到的 feature map 中,source 和 target 的同一类型的数据应当映射到一起。

\subsection{分析}

\subsubsection{部分特征的选择}

从最简单的形式去考虑,特征选择向量是一个01串,用来表示是否选择某个特征,而实际情况下对于某个特征是否选择不能一刀切,而是用一定的权重去表示其重要性,或者说某一个特征维度与 target 的相关性。

因此用一个 NN 的结构去学习这个特征筛选向量。

在 feature map 上,某个 source 特征维度与 target 的相关性可以用其与 target 的距离来体现,由于是按维度计算差异,所以使用绝对值差异:

$$
dist = |f_s - f_t|
$$

其中 $f_s, f_t$ 分别为 source 和 target 的特征。

加权向量 $v$ 则由 NN (暂时称作PAN)学习得到:

$$
v = PAN(dist)
$$

那么提纯后的特征 $f_{s,ex} = f_s v$,对于每一个source domain 都生成一个 $v$,对于 $f_t$ 则不需要用 $v$ 加权。

为了防止 $v$ 出现零解或者无穷解,用规则化限制 $v$,其中 $v$ 由所有 source domain 的平均值表示:

$$
R(v) = \sum(1 - \bar{v})^2
$$

用高阶矩来作为 PAN 的 Loss:

$$
L_{PAN}(\mathcal{D}_S, \mathcal{D}_T) = \sum_i(\|\mathbb{E}(f_{si,ex}^k) - \mathbb{E}({f_t}^k)\|_2) + \sum_i\sum_j(\|\mathbb{E}(f_{si,ex}^k) - \mathbb{E}(f_{sj,ex}^k)\|_2)
$$

\subsubsection{同类聚集,不同类排斥}

相同类别的数据,用平方和当作 loss,定义类别的中心 $f_c^1 = \frac{1}{N}\sum(f_s^1)$,用 moving average 去维护:

$$
f_c^i = \beta_{center} f_c^{i-1} + (1-\beta_{center}) \frac{1}{N}\sum(f_s^i)
$$

其中 $i$ 为 batch 的编号,$N$ 表示某个 batch 中对应 class 的样本个数,每一个 domain 的每一个 class 对应一个 center。

计算类别内数据与中心的距离作为 class Loss:

$$
L_{class} = \sum_i(\sum_j(f_c - f_s)^2)
$$

其中 $i$ 表示 domain 编号,j 表示类别编号。

还需要定义一个 domain 之间的 loss,用于将 source 与 target 拉近:

$$
L_{domain} = \sum_i(f_t - f_{ci})^2
$$

其中 $i$ 表示 source domain 的编号。

对于 target 的不同类别的数据,以类别的中心作为代表,计算如下 Loss,其中 $x$ 表示 feature:

$$
r_{ij} = \sqrt{\sum(x_i - x_j)}
$$

$$
L_{dis} = \sum_i\sum_j(B - r_{ij})^2
$$

其中 $B$ 为超参,用于设置不同类别中心点的离散程度,限制不同类别的数据中心点不能相隔太近,
并且不会距离太远导致无穷解。

\subsubsection{使用 threshold 选择性学习}

设置 threshold 的时候应选取上界。

对于每个 class 分别设置 threshold。

source 与 target 的中心点之间的欧氏距离作为 domain threshold,source 内部的数据与 class 中心点的距离作为 class threshold,
两个 threshold 都用 moving average 去更新。

\section{实验}

目前已经实现了除 threshold 以外的代码并跑通,现在包含四个 loss 的模型还在服务器上训练。

\section{疑问/困难}

\begin{enumerate}
    \item 如果在线进行数据蒸馏(设置 threshold 的方法),需要在一个 batch 中,只有一部分数据对模型的参数更新产生作用,目前还没有想好该怎么实现。
\end{enumerate}

\end{document}
